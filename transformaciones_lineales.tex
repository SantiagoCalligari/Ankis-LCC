\documentclass[a4paper]{article}

% PACKAGES
\usepackage[
    a4paper,
    margin=5mm % Small margin around the page (adjust if needed)
]{geometry} % Package to control page layout and margins
\usepackage{tikz} % Powerful drawing package, used here for the grid
\usetikzlibrary{calc} % To perform calculations within tikz
\usepackage{lipsum} % Package to generate placeholder text (Lorem Ipsum) - Keep for example
\usepackage{amsfonts} % Or \usepackage{amssymb}

% SETTINGS
\pagestyle{empty} % Remove page numbers
\setlength{\parindent}{0pt} % Remove paragraph indentation

\begin{document}

% --- PAGE 1 ---
\begin{tikzpicture}[remember picture, overlay]
    % Define the usable area corners
    \coordinate (TopLeft) at ($(current page.north west) + (5mm,-5mm)$);
    \coordinate (BottomRight) at ($(current page.south east) + (-5mm,5mm)$);

    % Calculate the dimensions of each mini-page
    \pgfmathsetmacro{\MiniPageW}{\textwidth/2}
    \pgfmathsetmacro{\MiniPageH}{\textheight/4}
    \pgfmathsetmacro{\TextW}{\MiniPageW - 4mm} % Text width slightly smaller than cell

    % --- Define Node Style ---
    \tikzstyle{minipagestyle} = [
        draw=gray, thin,
        anchor=north west,
        minimum width=\MiniPageW pt,
        minimum height=\MiniPageH pt,
        text width=\TextW pt,
        inner sep=2mm % Padding inside the cell
    ]

    % --- Draw the 8 Mini-Pages for Page 1 ---

    % Row 1
    \node[minipagestyle, text centered] at ($(TopLeft) + (0*\MiniPageW pt, -0*\MiniPageH pt)$)
    { % Content for R1, C1 (Top-Left)
        \vspace*{\fill} % Optional: centers text vertically
          Transformaciones Lineales
        \vspace*{\fill}
    };
    \node[minipagestyle, text centered] at ($(TopLeft) + (1*\MiniPageW pt, -0*\MiniPageH pt)$)
    { % Content for R1, C2 (Top-Right)
        \vspace*{\fill}
          Caracteristica de Transformaciones Lineales
        \vspace*{\fill}
    };

    % Row 2
    \node[minipagestyle, text centered] at ($(TopLeft) + (0*\MiniPageW pt, -1*\MiniPageH pt)$)
    { % Content for R2, C1
        \vspace*{\fill}
          Nucleo e Imagen de Transformaciones Lineales
        \vspace*{\fill}
    };
    \node[minipagestyle, text centered] at ($(TopLeft) + (1*\MiniPageW pt, -1*\MiniPageH pt)$)
    { % Content for R2, C2
        \vspace*{\fill}
          Espacio $L(V,W)$
        \vspace*{\fill}
    };

    % Row 3
    \node[minipagestyle, text centered] at ($(TopLeft) + (0*\MiniPageW pt, -2*\MiniPageH pt)$)
    { % Content for R3, C1
        \vspace*{\fill}
          Operadores Lineales
        \vspace*{\fill}
    };
    \node[minipagestyle, text centered] at ($(TopLeft) + (1*\MiniPageW pt, -2*\MiniPageH pt)$)
    { % Content for R3, C2
        \vspace*{\fill}
        Title Placeholder 6
        \vspace*{\fill}
    };

    % Row 4
    \node[minipagestyle, text centered] at ($(TopLeft) + (0*\MiniPageW pt, -3*\MiniPageH pt)$)
    { % Content for R4, C1 (Bottom-Left)
        \vspace*{\fill}
        Title Placeholder 7
        \vspace*{\fill}
    };
    \node[minipagestyle, text centered] at ($(TopLeft) + (1*\MiniPageW pt, -3*\MiniPageH pt)$)
    { % Content for R4, C2 (Bottom-Right)
        \vspace*{\fill}
        Title Placeholder 8
        \vspace*{\fill}
    };

\end{tikzpicture}

\newpage % Start the second page

% --- PAGE 2 ---
\begin{tikzpicture}[remember picture, overlay]
    % Define the usable area corners (same as page 1)
    \coordinate (TopLeft) at ($(current page.north west) + (5mm,-5mm)$);
    \coordinate (BottomRight) at ($(current page.south east) + (-5mm,5mm)$);

    % Calculate the dimensions of each mini-page (same as page 1)
    \pgfmathsetmacro{\MiniPageW}{\textwidth/2}
    \pgfmathsetmacro{\MiniPageH}{\textheight/4}
    \pgfmathsetmacro{\TextW}{\MiniPageW - 4mm} % Text width slightly smaller than cell

    % --- Use the same Node Style ---
    \tikzstyle{minipagestyle} = [
        draw=gray, thin,
        anchor=north west,
        minimum width=\MiniPageW pt,
        minimum height=\MiniPageH pt,
        text width=\TextW pt,
        inner sep=2mm % Padding inside the cell
    ]

    % --- Draw the 8 Mini-Pages for Page 2 ---

    % Row 1
    \node[minipagestyle] at ($(TopLeft) + (0*\MiniPageW pt, -0*\MiniPageH pt)$)
    { % Content for R1, C1 (Top-Left)
      Dado $T: V\rightarrow W$ Transformacion Lineal T es:
      \vspace{10mm}
      \begin{itemize}
        \footnotesize
        \item Un monomorfismo si T es inyectiva. [ $x \neq y \Rightarrow f(x) \neq f(y)$ ]
        \item Un epimorfismo si T es sobreyectiva. [ $\forall y, \exists x$ / $f(x) = y$ ]
        \item Un isomorfismo si T es biyectiva. [ inyectiva y sobreyectiva ]
        \item Un endomorfismo si T es $V = W$.
        \item Un automorfismo si T es $V = W$ y T es biyectiva.
        \item T es automorfismo $\Leftrightarrow$ T es isomorfismo y endomorfismo
      \end{itemize}
    };
    \node[minipagestyle] at ($(TopLeft) + (1*\MiniPageW pt, -0*\MiniPageH pt)$)
    { % Content for R1, C2 (Top-Right)
      (V, +, ·) y (W, +, ·) $\mathbb{F}$-ev, $T: V \rightarrow W$ T.L. si:
      \begin{itemize}
        \footnotesize
        \item $T(u + v) = T(u) + T(v)$
        \item $T(\alpha u) = \alpha T(u)$
        \item El cuerpo de los dos espacios es el MISMO
        \item T T.L. $\Rightarrow T(\bar{0}_V) = \bar{0}_W.$
      \item T T.L. $\Leftrightarrow \forall \alpha, \beta \in \mathbb{F}, \forall u,v \in V, T(\alpha \cdot u + \beta \cdot v )$
      \item T T.L. $\Rightarrow T(\sum_{i=1}^n \alpha_i \cdot u_i) = \sum_{i=1}^n \alpha_i T(u_i)$
      \item $A \in F^{mxn}, T_A: F^n \rightarrow F^m / T_A(x) = Ax$ T.L.
      \item Si $S \subset V$ sev $\Rightarrow T(S) \subset W$ sev.
      \item Si $R \subset W$ sev $\Rightarrow T^-1(R) \subset V$ sev.
      \item $B = {v_1,.., v_n}$ base y $w_1,..,w_n \in W \Rightarrow \exists! T / T(v_i) = w_i, i = 1,..,n$
      \end{itemize}

    };

    % Row 2
    \node[minipagestyle] at ($(TopLeft) + (0*\MiniPageW pt, -1*\MiniPageH pt)$)
    { % Content for R2, C1
      V,W F-ev, $T,S$:$ V\rightarrow W$ T.L. $\lambda \in F \Rightarrow$
      \begin{itemize}
        \item $(T+S)(v) = T(v) + S(v)$ T.L.
        \item $(\lambda T)(v) = \lambda T(v)$ T.L.
        \item $L(V,W) = \{T:V\rightarrow W / T$ es una T.L.\}
        \item ((L(V,W), +, ·) es un F Espacio Vectorial.
        \item $dimV = n$, $dimW = m \Rightarrow dim(L(V,W)) = m\cdot n$
      \item $T\in L(U,V), S \in L(V,W) \Rightarrow S$\textopenbullet$T \in L(U,W)$


      \end{itemize}
    };
    \node[minipagestyle] at ($(TopLeft) + (1*\MiniPageW pt, -1*\MiniPageH pt)$)
    { % Content for R2, C2
      $T:V\rightarrow W$ T.L.
      \begin{itemize}
        \scriptsize
        \item $ker(T)=T^{-1}(\bar{0})=\{v\in V:T(v) = \bar{0}\}$ subeespacio de V.
        \item T monomorfismo $\Leftrightarrow ker(T) = \{\bar{0}\}$.
        \item $Im(T) = T(V) = \{ T(v) \in W: v \in V\} $ subespacio de W.
        \item T epimorfismo $\Leftrightarrow Im(T) = W$
        \item Nulidad de T es\hspace{.5mm}la\hspace{.5mm}dimension\hspace{.5mm}de su nucleo.\hspace{1mm}$nul(T) = dim(ker(T)) $.
        \item Rango de T\hspace{.6mm}es\hspace{.6mm}la\hspace{.6mm}dimension\hspace{.6mm}de su imagen.\hspace{1mm}$ran(T) = dim(Im(T))$.
        \item T monomorfismo $\Leftrightarrow nul(T) = 0$. Epimorfimo $\Leftrightarrow ran(T) = dim(W)$.
        \item T. De la dimension: si V finito dim. $ran(T) + nul(T) = dim(V)$
        \item Si V finito dim. y $dim(V) > dim(W) \Rightarrow T$ no monomorfismo. 
        \item Si V finito dim. y $dim(V) < dim(W) \Rightarrow T$ no epimorfismo. 

      \end{itemize}
    };

    % Row 3
    \node[minipagestyle] at ($(TopLeft) + (0*\MiniPageW pt, -2*\MiniPageH pt)$)
    { % Content for R3, C1
        Section 5: \lipsum[4]
    };
    \node[minipagestyle] at ($(TopLeft) + (1*\MiniPageW pt, -2*\MiniPageH pt)$)
    { % Content for R3, C2
      Cuando V = W, se escribe L(V). $T, S, R, L \in L(V), \lambda \in F$
      \begin{itemize}
        \scriptsize
        \item Si $S,T \in L(V) \Rightarrow S$\textopenbullet$T \in L(V)$. $ST := S$\textopenbullet$T$. 
        \item $T(SR) = (TS)R$.
        \item $id_VT = Tid_V = T$
        \item $T(S+R) = TS + TR$ y $ (S + R)T = ST + SR$.
        \item $\lambda(TS) = (\lambda T)S = T(\lambda S)$
        \item T es invertible $\Leftrightarrow \exists S: W\rightarrow V / TS = id_W$ y $ST = id_V$.
        \item Si existe inversa es unica. T invertible $\Leftrightarrow$ biyectiva.
        \item Si $T \in L(V,W)$ invertible $\Rightarrow  T^{-1} \in L(W,V)$ 
        \item si $T\in L(U,V), S\in L(V,W)$ invertibles $\Rightarrow ST\in L(U,W)$ invertibles
          y $(ST)^{-1} = T^{-1}S^{-1} \in L(W,U)$

        \end{itemize}
    };

    % Row 4
    \node[minipagestyle] at ($(TopLeft) + (0*\MiniPageW pt, -3*\MiniPageH pt)$)
    { % Content for R4, C1 (Bottom-Left)
        Section 7: \lipsum[6]
    };
    \node[minipagestyle] at ($(TopLeft) + (1*\MiniPageW pt, -3*\MiniPageH pt)$)
    { % Content for R4, C2 (Bottom-Right)
        Section 8: \lipsum[7]
    };

\end{tikzpicture}

\end{document}

